\chapter{Metodologia}
Per explicar la part de metodologia, hauré d'explicar molta informació relacionada amb la informàtica, i intentaré explicar-ho tot de manera que es pugui entendre. Després d’aquesta explicació informàtica, procediré a explicar els materials que utilitzo en els experiments de la part pràctica d’aquest treball.
\vspace{0.3truecm}

Per explicar la part de metodologia, hauré d’explicar molta informació relacionada amb la informàtica, i intentaré explicar-ho tot de manera que es pugui entendre. Després d’aquesta explicació informàtica, procediré a explicar els materials que utilitzo en els experiments de la part pràctica d’aquest treball.\\

Explicaré la informació de manera clara; el primer punt que explicaré és sobre les màquines virtuals.

\section{Màquina virtual}
\subsection{Què és una màquina virtual (VM)?}
Una màquina virtual és un entorn informàtic que funciona com un sistema aïllat amb la seva pròpia CPU (Unitat Central de Processament), memòria, interfície de xarxa i emmagatzematge, el qual es crea a partir del hardware.

\textit{\textbf{Hardware:}} Són tots els components físics d’un sistema informàtic, és a dir, les parts tangibles que es poden veure i tocar.

Les màquines virtuals utilitzen sotware en un ordinador físic (host) per replicar o emular la funcionalitat d’un ordinador diferent. En resum, una màquina virtual és un ordinador simulat dins d’un ordinador real.

Les màquines virtuals funcionen igual que els ordinadors normals: tenen un sistema operatiu, emmagatzemen fitxers, executen programes, etc.

\subsection{Tipus de màquines virtuals}
Les VM poden tenir diferents tasques en funció del tipus de màquina virtual utilitzada.
\begin{enumerate}
 \item \textbf{Màquina virtual de procés}: Serveix per executar un programa com si fos natiu, sense importar el sistema operatiu o el hardware real. No instal·la un sistema operatiu complet, sinó que permet executar programes d’una altra plataforma.

 \textbf{Avantatges:}
 \begin{enumerate}[1)]
  \item Consumeix pocs recursos (RAM i CPU) comparat amb un sistema operatiu complet.
  \item Pots executar programes en qualsevol sistema operatiu compatible.
  \item Si falla, no afecta el sistema operatiu principal.
 \end{enumerate}
 \textbf{Desavantatges:}
 \begin{enumerate}[1)]
  \item Només funciona per a un tipus de programa o plataforma específica (ex. Java).
  \item No pots executar un sistema operatiu complet, només aplicacions.
  \item Accés limitat al sotware: no pot utilitzar tot el potencial del PC.
 \end{enumerate}

 \item \textbf{Màquina virtual de sistema}: Serveix per emular un sistema operatiu complet dins d’un altre; per exemple, pots tenir Linux dins de Windows, o Windows dins de Linux.

  \textbf{Avantatges:}
 \begin{enumerate}[1)]
  \item Permet executar un sistema operatiu complet dins d’un altre.
  \item Pots provar diferents sistemes operatius sense tocar el PC real.
  \item Cada màquina virtual està aïllada, així que errors o virus no afecten el sistema principal.
 \end{enumerate}
 \textbf{Desavantatges:}
 \begin{enumerate}[1)]
  \item Consumeix molts recursos: necessita RAM, CPU i espai al disc.
  \item Pot ser més lenta que usar un sistema operatiu natiu.
  \item Configurar i mantenir diverses màquines virtuals pot ser més complex.
 \end{enumerate}
\end{enumerate}

\subsection{Quina he utilitzat jo?}
Jo no he utilitzat cap maquina virtual, l'eina que he utilitzada és un \textbf{sistema operatiu}.

\section{Sistema operatiu}
\subsection{Què és un sistema operatiu i quina difèrencia hi ha amb la maquina virtual?}
`Els sistemes operatius (tambè anomenats nuclis o kernels) són un conjunt de programes informatics que fa de \textbf{cervell del teu ordinador}, gestionant els seus recursos físics (hardware), com el processador, la memòria i els perifèrics, i els programes (software), per permetre l'execució d'aplicacions i facilitar la interacció entre l'usuari i la màquina.`  \textit{\textbf{Fonts:}} \href{https://www.ejemplos.co/20-ejemplos-de-sistemas-operativos/#:~:text=Un\%20Sistema\%20Operativo\%20(SO)\%20es,\%2C\%20MacOS\%2C\%20Windows\%2C\%20Haiku.}{Enciclopedia de ejemplos}
\subsection{Avantatges i Desavantatges}
Els sistemes operatius tenen moltes avantatges, i una de les més importants a destacar és la \textbf{compatibilitat amb pràcticament tot el hardware}. Això és degut al fet que la majoria d’usuaris utilitzen Windows, i per aquest motiu la majoria de proveïdors de hardware fabriquen controladors per a Windows.

També cal destacar la \textbf{facilitat d’ús} i el \textbf{suport de software}. Aquest últim punt és perquè Windows té una gran audiència, de manera que els desenvolupadors prefereixen crear jocs i software per a aquest sistema operatiu, donant-li una major optimització.\\

No només parlaré dels punts positius, sinó també algunes desavantatges que pot tenir, com ara els \textbf{atacs de virus elevats}. La raó principal és que els `pirates informàtics` poden trencar fàcilment la seguretat de Windows, per la qual cosa els usuaris d’aquest sistema depenen del software antivirus, que requereix pagaments mensuals per protegir-se.

Una altra desavantatge és \textbf{l’alt consum de recursos informàtics}. Per exemple, si estàs instal·lant el sistema operatiu Windows, l’ordinador ha de tenir una gran capacitat de RAM, una targeta gràfica potent i molt d’espai al disc dur.
\textit{\textbf{Fonts:}} \href{https://aslenovatec.com/reviews/ventajas-y-desventajas-del-sistema-operativo-windows/}{AslenovaTec}\\

Hi ha moltes més avantatges i desavantatges, però no les puc anomenar totes, ja que això donaria per fer un altre treball de recerca.

\subsection{Quin sistema operatiu he utilizat per el treball?}
Per aquest treball he utilitzat el sistema operatiu \textit{Xubuntu}\cite{xubuntu}.
\section{Xubuntu}
\subsection{Què és Xubuntu?}
L’\textbf{Ubuntu} és un sistema operatiu basat en \textit{Linux}.\\

`\textbf{Linux} és un sistema operatiu de codi obert creat per Linus Torvalds l’any 1991, que funciona com una alternativa gratuïta i modificable a sistemes com Windows i macOS`
\href{https://www.redhat.com/es/topics/linux}{RedHat}\\


Xubuntu va ser llançat a l’octubre de 2004 per Canonical. Està basat en Linux i deriva de Debian (vegeu informació de Debian). Ubuntu és de codi obert, cosa que significa que tant el sistema com les seves aplicacions estan disponibles per ser estudiades sense cap cost.

\subsection{Per a què serveix Xubuntu?}
Hi ha una infinitat de propòsits, però els que més destaquen són els que descriuré a continuació:
\begin{enumerate}
 \item \textbf{Navegar per internet:} Inclou navegadors com Firefox, optimitzat.
 \item \textbf{Programar i desenvolupar software:} És molt utilitzat per desenvolupadors gràcies a la seva compatibilitat amb una infinitat d’eines de desenvolupament.
 \item \textbf{Utilitzar aplicacions d’oficina:} LibreOffice, Thunderbird o GIMP estan disponibles i són fàcils d’instal·lar.
 \item \textbf{Servidors web i bases de dades:} Ubuntu és la distribució de Linux més popular per a hosting i servidors al núvol.
 \item \textbf{Educació i tasques acadèmiques:} Moltes institucions i escoles l’empren per la seva gratuïtat.
 \item \textbf{Substituir Windows en ordinadors antics:} Consumeix pocs recursos comparat amb el SO de Microsoft, cosa que permet prolongar la vida útil del hardware.
\end{enumerate}

\textit{\textbf{Fonts:}} \href{https://www.godaddy.com/resources/es/crearweb/que-es-ubuntu-y-para-que-sirve}{Equipo de Contenidos de GoDaddy}

\subsection{Per què vaig escollir Xubuntu?}
La principal raó és perquè Ubuntu és gratuït i de codi obert; això fa que sigui accessible per a tothom. Té una gran comunitat, cosa que fa que sigui fàcil trobar ajuda i, per últim, perquè és fàcil d’utilitzar per a principiants.

\section{LaTeX}
LaTeX és un sistema de composició de text orientat a la creació de documents escrits que presentin una alta qualitat tipogràfica. Però, com funciona aquesta eina i quina utilitat té?

És una eina per crear documents d’accés lliure i gratuïts que ja s’utilitza àmpliament en molts sectors de la societat. En aquesta eina l’autor s’ha de centrar només en el contingut del que escriu, en lloc de la presentació visual. L’autor especificarà l’estructura lògica utilitzant conceptes familiars com: capítol, secció, taula, figura, etc., deixant que el sistema LaTeX s’ocupi de la presentació visual d’aquestes estructures.
\textit{\textbf{Fonts:}} \href{https://computerhoy.20minutos.es/tecnologia/latex-como-funciona-util-herramienta-crear-documentos-1165366#3-1670263544507}{ComputerHoy}\\
\subsection{Avantatges i Desavantatges}
\subsection{Per què vaig decidir aprendre LaTeX?}
Hi ha molts arguments del perquè vaig decidir aprendre i utilitzar LaTeX. Un dels motius principals va ser gràcies al meu tutor del treball, Fernando, que ens va parlar dels molts avantatges que té, i que és una eina estàndard en l’àmbit acadèmic i científic. Aprendre-la ara ens ajudarà molt en el futur, quan haguem de fer més treballs científics. A més, gestiona de manera senzilla fórmules matemàtiques, bibliografies i referències, i vaig considerar que m’ajudaria en el meu desenvolupament acadèmic i professional.\\

Para facilitar mi uso en LaTeX, utilize el editor de Kile. \ref{sec:Kile}
\section{Kile} \label{sec:Kile}

\section{GitHub}



