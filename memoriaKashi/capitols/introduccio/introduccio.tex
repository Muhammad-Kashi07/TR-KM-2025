\chapter{Introducció}

L’aigua és un recurs fonamental per a la vida. Cada dia la fem servir per beure, cuinar, netejar o regar, i sovint donem per fet que sempre serà accessible i segura. Tanmateix, la seva qualitat no és un aspecte trivial: segons dades de l’Organització Mundial de la Salut \cite{OrgaMS}, més de 2.000 milions de persones al món consumeixen aigua contaminada, la qual cosa pot provocar greus problemes de salut i afectar els ecosistemes aquàtics.

Aquest treball de recerca (TR) es dividirà en 3 \textbf{capítols essencials}:

\begin{itemize}
  \item En el \textbf{Capítol 4}~\nameref{c:Metodologia}, em centraré en explicar les eines que he utilitzat per poder completar correctament aquest treball, així com les raons del seu ús.

  \item El \textbf{Capítol 3}~\nameref{c:RecercaPrevia} es centra en l’estudi de diversos paràmetres físics, químics i biològics que permeten determinar la qualitat de l’aigua. A través d’aquests indicadors (com el pH, la duresa, la presència de nitrats, el clor, la temperatura o la presència de microorganismes) és possible entendre millor fins a quin punt l’aigua és apta per al consum humà i per al manteniment dels ecosistemes.

  \item Finalment, en el \textbf{Capítol 5}~\nameref{c:Partpra}, desenvoluparé la part pràctica, on realitzaré un total de tres experiments amb tota la informació obtinguda de la recerca prèvia, per poder calcular la qualitat de l’aigua. Els resultats d’aquests experiments seran recollits i analitzats a les conclusions del treball. (\textbf{PON REFERENCIAS})
\end{itemize}

\clearpage

\textbf{EN ÀNGLES / IN ENGLISH}

Water is a fundamental resource for life. Every day we use it to drink, cook, clean, or water plants, and we often take for granted that it will always be accessible and safe. However, its quality is not a trivial matter: according to data from the World Health Organization \cite{OrgaMS}, more than 2 billion people around the world consume contaminated water, which can cause serious health problems and affect aquatic ecosystems.

This research project (TR) will be divided into 3 \textbf{essential chapters}:

\begin{itemize}
\item In \textbf{Chapter 4}~\nameref{c:Metodologia}, I will focus on explaining the tools I have used to correctly complete this project, as well as the reasons for their use.

\item \textbf{Chapter 3}~\nameref{c:RecercaPrevia} focuses on the study of various physical, chemical, and biological parameters that make it possible to determine water quality. Through these indicators (such as pH, hardness, the presence of nitrates, chlorine, temperature, or the presence of microorganisms), it is possible to better understand to what extent water is suitable for human consumption and for the maintenance of ecosystems.

\item Finally, in \textbf{Chapter 5}~\nameref{c:Partpra}, I will develop the practical part, where I will carry out a total of three experiments using all the information obtained from the previous research, in order to calculate water quality. The results of these experiments will be collected and analyzed in the conclusions of the project. (\textbf{ADD REFERENCES})
\end{itemize}

