\chapter{Metodologia}
Per explicar la part de metodologia, hauré d'explicar molta informació relacionada amb la informàtica, i intentaré explicar-ho tot de manera que es pugui entendre. Després d’aquesta explicació informàtica, procediré a explicar els materials que utilitzo en els experiments de la part pràctica d’aquest treball.
\vspace{0.3truecm}

Per explicar la part de metodologia, hauré d’explicar molta informació relacionada amb la informàtica, i intentaré explicar-ho tot de manera que es pugui entendre. Després d’aquesta explicació informàtica, procediré a explicar els materials que utilitzo en els experiments de la part pràctica d’aquest treball.\\

Explicaré la informació de manera clara; el primer punt que explicaré és sobre les màquines virtuals.

\section{Màquina virtual}
\subsection{Què és una màquina virtual (VM)?}
Una màquina virtual és un entorn informàtic que funciona com un sistema aïllat amb la seva pròpia CPU (Unitat Central de Processament), memòria, interfície de xarxa i emmagatzematge, el qual es crea a partir del hardware.

\textit{\textbf{Hardware:}} Són tots els components físics d’un sistema informàtic, és a dir, les parts tangibles que es poden veure i tocar.

Les màquines virtuals utilitzen sotware en un ordinador físic (host) per replicar o emular la funcionalitat d’un ordinador diferent. En resum, una màquina virtual és un ordinador simulat dins d’un ordinador real.

Les màquines virtuals funcionen igual que els ordinadors normals: tenen un sistema operatiu, emmagatzemen fitxers, executen programes, etc.

\subsection{Tipus de màquines virtuals}
Les VM poden tenir diferents tasques en funció del tipus de màquina virtual utilitzada.
\begin{enumerate}
 \item \textbf{Màquina virtual de procés}: Serveix per executar un programa com si fos natiu, sense importar el sistema operatiu o el hardware real. No instal·la un sistema operatiu complet, sinó que permet executar programes d’una altra plataforma.

 \textbf{Avantatges:}
 \begin{enumerate}[1)]
  \item Consumeix pocs recursos (RAM i CPU) comparat amb un sistema operatiu complet.
  \item Pots executar programes en qualsevol sistema operatiu compatible.
  \item Si falla, no afecta el sistema operatiu principal.
 \end{enumerate}
 \textbf{Desavantatges:}
 \begin{enumerate}[1)]
  \item Només funciona per a un tipus de programa o plataforma específica (ex. Java).
  \item No pots executar un sistema operatiu complet, només aplicacions.
  \item Accés limitat al sotware: no pot utilitzar tot el potencial del PC.
 \end{enumerate}

 \item \textbf{Màquina virtual de sistema}: Serveix per emular un sistema operatiu complet dins d’un altre; per exemple, pots tenir Linux dins de Windows, o Windows dins de Linux.

  \textbf{Avantatges:}
 \begin{enumerate}[1)]
  \item Permet executar un sistema operatiu complet dins d’un altre.
  \item Pots provar diferents sistemes operatius sense tocar el PC real.
  \item Cada màquina virtual està aïllada, així que errors o virus no afecten el sistema principal.
 \end{enumerate}
 \textbf{Desavantatges:}
 \begin{enumerate}[1)]
  \item Consumeix molts recursos: necessita RAM, CPU i espai al disc.
  \item Pot ser més lenta que usar un sistema operatiu natiu.
  \item Configurar i mantenir diverses màquines virtuals pot ser més complex.
 \end{enumerate}
\end{enumerate}

\subsection{Quina he utilitzat jo?}
Dels dos tipus de màquines virtuals que hi ha, jo vaig triar la màquina virtual de sistema. La veritat és que em sembla molt millor que la màquina virtual de procés, tot i que això no significa que sigui objectivament millor. Però les avantatges que té la màquina virtual de sistema em semblen més grans que les del procés, encara que, al cap i a la fi, les dues són màquines virtuals.

Per fer aquest treball, amb l’ajuda del meu tutor Fernando Garcia, vaig poder instal·lar una màquina virtual de sistema i, dins d’aquesta màquina, vaig instal·lar el sistema operatiu Xbuntu.


\section{Xubuntu}
Per aquest treball he utilitzat el sistema operatiu \textit{Xubuntu}\cite{xubuntu}.
\subsection{Què és Xubuntu?}
L’\textbf{Ubuntu} és un sistema operatiu basat en \textit{Linux} (la informació sobre Linux està explicada en una altra secció).

Abans d’explicar més sobre Xubuntu, hem de fer la següent pregunta: què és un sistema operatiu? \\

Un sistema operatiu (SO) és un programa o un conjunt de programes que actua com a intermediari entre el hardware d’un ordinador i les aplicacions que utilitzem, permetent que aquestes s’executin i gestionant els recursos del sistema. (\textit{\textbf{Font:}} \href{https://desarrollarinclusion.cilsa.org/tecnologia-inclusiva/que-es-un-sistema-operativo/#:~:text=Un\%20sistema\%20operativo\%20es\%20un,placa\%20de\%20red\%2C\%20entre\%20otros.}{Desarrollar Inclusión}) \\

Continuant amb l’explicació de Xubuntu: Xubuntu és un SO gratuït que va ser llançat a l’octubre de 2004 per Canonical. Està basat en Linux i deriva de Debian (vegeu informació de Debian). Ubuntu és de codi obert, cosa que significa que tant el sistema com les seves aplicacions estan disponibles per ser estudiades sense cap cost.

\subsection{Per a què serveix Xubuntu?}
Hi ha una infinitat de propòsits, però els que més destaquen són els que descriuré a continuació:
\begin{enumerate}
 \item \textbf{Navegar per internet:} Inclou navegadors com Firefox, optimitzat.
 \item \textbf{Programar i desenvolupar software:} És molt utilitzat per desenvolupadors gràcies a la seva compatibilitat amb una infinitat d’eines de desenvolupament.
 \item \textbf{Utilitzar aplicacions d’oficina:} LibreOffice, Thunderbird o GIMP estan disponibles i són fàcils d’instal·lar.
 \item \textbf{Servidors web i bases de dades:} Ubuntu és la distribució de Linux més popular per a hosting i servidors al núvol.
 \item \textbf{Educació i tasques acadèmiques:} Moltes institucions i escoles l’empren per la seva gratuïtat.
 \item \textbf{Substituir Windows en ordinadors antics:} Consumeix pocs recursos comparat amb el SO de Microsoft, cosa que permet prolongar la vida útil del hardware.
\end{enumerate}

\textit{\textbf{Fonts:}} \href{https://www.godaddy.com/resources/es/crearweb/que-es-ubuntu-y-para-que-sirve}{Equipo de Contenidos de GoDaddy}

\subsection{Per què vaig escollir Xubuntu?}
La principal raó és perquè Ubuntu és gratuït i de codi obert; això fa que sigui accessible per a tothom. Té una gran comunitat, cosa que fa que sigui fàcil trobar ajuda i, per últim, perquè és fàcil d’utilitzar per a principiants.





