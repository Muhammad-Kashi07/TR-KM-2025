\chapter{Conclusions}
\label{c:conclusions}

% Aquest treball de recerca m'ha permès abordar l'estudi de la qualitat de l’aigua des de diverses perspectives: teòrica, pràctica i personal. A nivell científic, hem pogut identificar i comprendre els principals paràmetres que defineixen la qualitat de l’aigua, aprendre a mesurar-los amb eines senzilles com les tires reactives i interpretar els resultats obtinguts amb rigor.
%
% La part pràctica m'ha posat a prova els coneixements previs i les habilitats adquirides, permetent-me experimentar de manera controlada i analitzar les dades obtingudes. Els experiments han evidenciat la importància de seguir correctament els procediments per obtenir mesures fiables i han generat recomanacions útils per a futurs treballs similars.
%
% A més, aquest projecte ha tingut un gran valor formatiu. He après a cercar informació de manera autònoma, a utilitzar eines professionals de recerca i documentació com Linux, GitHub i LaTeX, i a organitzar el treball de manera eficient. Tot això ha contribuït al desenvolupament de competències útils tant per a futurs estudis universitaris com per a altres projectes científics.
%
% Finalment, aquest treball no només ens ha ajudat a comprendre millor la qualitat de l’aigua, sinó que també ha servit per experimentar un procés de recerca complet: plantejar un problema, dissenyar experiments, analitzar dades i extreure conclusions. Hem pogut desenvolupar una visió més crítica i rigorosa del mètode científic, així com valorar la importància de la investigació accessible i clara per a qualsevol persona interessada en el tema.

Aquest treball en l'àmbit de les ciències experimentals m'ha permès extreure diverses conclusions. No només he obtingut conclusions sobre els resultats dels experiments, sinó que també he tret conclusions de la part de la metodologia i de la recerca prèvia.

\section{Conclusions de la metodologia}

Durant aquesta etapa, hem aprés a buscar i seleccionar informació científica de manera crítica, així com a organitzar-la de forma estructurada. L’ús d’eines com Linux, GitHub i LaTeX ens ha permès treballar en un entorn col·laboratiu eficient, i la supervisió del tutor ha estat clau per orientar la recerca i assegurar la qualitat del contingut. En resum, la metodologia no només ha aportat coneixements tècnics, sinó que també ha reforçat les habilitats de treball en equip, planificació i organització.

\section{Conclusions de la recerca prèvia}

La recerca prèvia ha estat fonamental per establir les bases teòriques del treball. M'ha permès comprendre millor els conceptes clau relacionats amb la qualitat de l’aigua i identificar els paràmetres més rellevants per a la seva mesura. Aquest procés ha facilitat la presa de decisions en la fase pràctica, assegurant que els experiments estiguessin ben fonamentats i alineats amb els objectius del treball.



\section{Conclusions de la part pràctica}

La part pràctica ha permès aplicar els coneixements teòrics adquirits i desenvolupar competències experimentals. Hem pogut mesurar la qualitat de l’aigua amb tires reactives, interpretar les dades i extreure conclusions fiables. Els assajos diferenciats ens han ajudat a comprovar la variabilitat dels resultats i a formular recomanacions per obtenir mesures més consistents.

Aquest capítol ha evidenciat la importància de treballar en un entorn col·laboratiu, on la coordinació amb el tutor ha facilitat la resolució de dubtes i la correcta execució dels experiments. A més, l’experiència pràctica ha reforçat la capacitat de documentar processos, organitzar dades i comunicar els resultats de manera clara.

En resum, la part pràctica ha completat el procés d’aprenentatge: ha combinat rigor científic, aplicació de coneixements i desenvolupament d’habilitats personals i tècniques. Aquest aprenentatge ha preparat el camí per a futurs projectes científics, aportant experiència tant en recerca com en treball en equip, planificació i gestió de recursos limitats.

Els resultats de la part pràctica es podran observar millor a la secció de \nameref{s:resultats}, en el capítol~\ref{c:Partpra}~\nameref{c:Partpra}
\section{Conclusió final}
Amb aquest capítol dono per finalitzat el meu TR. Al llarg del projecte he combinat recerca teòrica, experiments pràctics i l’ús d’eines professionals per aprofundir en la qualitat de l’aigua, tot treballant amb la supervisió i l’ajuda del meu tutor.

Aquest procés m’ha permès desenvolupar habilitats de recerca, organització de dades, interpretació de resultats i comunicació científica, alhora que he après a gestionar recursos limitats i aplicar mètodes experimentals de manera rigorosa.

% A continuació, a l’apartat d’Annexos es poden consultar les taules dels experiments que he realitzat, i finalment es presenta la bibliografia que he utilitzat per fonamentar tot el treball.

