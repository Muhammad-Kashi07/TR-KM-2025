\chapter{Objectius}
\label{c:objectius}
L'objectiu principal del meu treball és fer recerca en el camp de la Química. Tot i les limitacions que tenim en recursos materials i humans (el meu tutor és professor de Matemàtiques), hem creat un grup de recerca tan real com ha estat possible.

En aquest grup de recerca el meu tutor ha actuat com el cap de recerca i ha fixat una sèrie de requisits propis del grup de treball. En canvi, jo he actuat com a investigador fixant el tema, fent la formació prèvia, etc .
A continuació detallarem els requisits fixats per cadascuna de les parts:

\subsubsection*{Requisits del grup de recerca}
\begin{itemize}
    \item Definir uns objectius clars i que es poguessin aconseguir dins del termini del TR.
    \item Marcar un calendari i organitzar bé les feines.
    \item Fer servir eines per generar transferència de coneixement i traçabilitat de la feina. Per tal d'assolir aquests objectius hem fet servir eines professionals en el camp de la recerca, com GitHub~\cite{GitHub}, que permeten compartir els documents i poder recuperar el procés de creació.
    \item Crear un treball completament lliure, és a dir, \textit{que qualsevol persona pot utilitzar, distribuir, copiar, estudiar, modificar, millorar i compartir perquè tothom se'n pugui beneficiar}~\cite{programariLliure}.
    \item Mantenir sempre un enfocament científic, encara que els recursos fossin limitats.
\end{itemize}

\subsubsection*{Requisits de l'investigador}
Com a investigador, jo també he fixat requisits que havia d'assolir el TR.
\begin{itemize}
    \item Escollir i concretar el tema (en aquest cas, la qualitat de l’aigua).
    \item Formar-me per a poder fer la recerca. En aquest cas, fer una recerca d'informació prèvia en llibres i articles de confiança.
    \item Aprendre a utilitzar les eines que em va proporcionar el tutor i posar-les en pràctica.
    \item Treballar amb totes les eines pròpies del grup de recerca (Linux, GitHub i LaTeX) de manera eficient per tal de crear una recerca de qualitat.
    \item Extreure les meves pròpies conclusions i explicar-ho tot de manera senzilla, perquè també pugui arribar a gent que no té coneixements en aquest camp.
\end{itemize}

Aquest treball no serà només llegir informació i treure conclusions, sinó també aprendre noves maneres de treballar, adaptar-se a eines de laboratori que no he utilitzat mai. Aprendre a fer-les servir correctament serà tot un procés, que s'explica amb més detall al capítol~\ref{c:Metodologia}~\nameref{c:Metodologia}.

\subsubsection{Objectiu formatiu}
Aquest treball també té com a objectiu la meva pròpia formació: aprendre a buscar informació pel meu compte, estructurar un treball científic pas a pas i dur a terme experiments propis. Tot aquest procés no només m’ha ajudat a fer aquest treball de recerca, sinó que també m’ha donat experiència i em prepara per a futurs reptes acadèmics i professionals.

\subsubsection{Objectiu de recerca}
L’objectiu de recerca o principal del meu treball és entendre millor la qualitat de l’aigua i fer-ho a partir d’un mètode senzill: l’ús de tires reactives. Per aconseguir-ho, abans he hagut d’informar-me bé sobre quins són els paràmetres que serveixen per mesurar aquesta qualitat, quins valors són normals i quins no, i també sobre alguns dels components químics i biològics que hi poden aparèixer.
% Aquesta part de recerca prèvia ha estat clau per tal de preparar la  part pràctica.



%%%%He puesto toda la informacion de abajo, en los subsecciones, mira a ver que tal esta la cosa%%%


%%%%% Aprofita el que tens per completar el que falta i si falta alguna cosa ho afegeixes.
%%%%%%%%%%%%%%%%%%%%%%%%%%%%%%%%%%%%%%%%%%%%%%%%%%%%%%%%%%%%%%%%%%%%%%%%%%
Per tal d'analitzar la qualitat de l’aigua mitjançant l’ús de tires reactives  de manera adequada ha estat necessari dur a terme una recerca prèvia sobre els paràmetres de qualitat de l’aigua, els valors mínims i màxims recomanats i alguns dels molts components químics i biològics que poden estar-hi presents. Aquesta fase teòrica ha permès adquirir una base sòlida per a l'experimentació.

Aquest treball no es limita a la recerca bibliogràfica i l’anàlisi de resultats, sinó que també ha implicat un procés d’aprenentatge personal. He hagut d’adaptar-me a l’ús de noves eines experimentals, fixades pel cap de recerca, i aprendre a utilitzar-les correctament.
% Aquest procés metodològic es descriu amb més detall al capítol de Metodologia (vegeu capítol~\nameref{c:Metodologia}).

% \subsubsection{Objectiu personal}
% Com a objectiu personal, vull que aquest treball no només tingui un interès acadèmic, sinó que també sigui capaç de resultar interesant per a persones que habitualment no s’interessen per temes científics. Fer entenedor un tema tècnic com la qualitat de l’aigua és un repte personal que vull assolir.
%
% Finalment, el treball de recerca també té un objectiu formatiu: aprendre a cercar informació de manera autònoma, a estructurar un treball científic i a realitzar una experimentació pròpia. Tot aquest procés em proporciona experiència i em prepara per a futurs reptes acadèmics i professionals.

