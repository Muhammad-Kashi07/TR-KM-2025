\chapter{Objectius}

L'objectiu principal del meu treball és fer recerca en el camp de la Química. Tot i les limitacions que tenim en recursos materials i humans (el meu tutor és professor de Matemàtiques), hem creat un grup de recerca tan real com ha estat possible.

En aquest grup de recerca el meu tutor ha actuat com el cap de recerca i ha fixat una sèrie de requisits propis del grup de treball. En canvi, jo he actuat com a investigador fixant el tema, fent la formació prèvia, ...
A continuació detallarem els requisits fixats per cadascuna de les parts:

\subsubsection*{Requisits del grup de recerca}
\begin{itemize}
    \item Definir uns objectius clars i que es poguessin aconseguir dins el temps que teníem.
    \item Marcar un calendari i organitzar bé les feines.
    \item Fer servir eines per treballar junts, com~\nameref{c:GH}., per compartir els documents i no perdre res pel camí.
    \item Mantenir sempre un enfocament científic, encara que els recursos fossin limitats.
\end{itemize}

\subsubsection*{Requisits de l'investigador}
Com a investigador, jo també tenia uns compromisos que havia de complir:
\begin{itemize}
    \item Escollir i concretar el tema (en aquest cas, la qualitat de l’aigua).
    \item Buscar informació prèvia en llibres i articles de confiança.
    \item Aprendre a utilitzar les eines que em va proporcionar el tutor i posar-les en pràctica.
    \item Treballar amb GitHub de manera activa per tenir totes les versions del treball organitzades.
    \item Fer les meves pròpies conclusions i intentar explicar-ho tot de manera senzilla, perquè també pugui interessar a gent que no és del món científic.
\end{itemize}

Aquest treball no ha serà només llegir informació i treure conclusions, sinó també aprendre noves maneres de treballar. Haure d’adaptar-me a eines de laboratori que no he utilitzat mai, i que em va proporcionar el meu tutor de TR, Fernando García. Aprendre a fer-les servir correctament serà tot un procés, i l’explico amb més detall al capítol de Metodologia (vegeu capítol~\nameref{c:Metodologia}).

\subsubsection{Objectiu formatiu}
A més, aquest treball m’ha servit per practicar coses que em seran útils més endavant: aprendre a buscar informació pel meu compte, estructurar un treball científic pas a pas i dur a terme experiments propis. Tot aquest procés no només m’ha ajudat a fer aquest projecte, sinó que també m’ha donat experiència i em prepara millor per a futurs reptes acadèmics i professionals.

\subsubsection{Objectiu principal}
L’objectiu central del meu treball és entendre millor la qualitat de l’aigua i fer-ho a partir d’un mètode senzill; l’ús de tires reactives. Per aconseguir-ho, abans he hagut d’informar-me bé sobre quins són els paràmetres que serveixen per mesurar aquesta qualitat, quins valors són normals i quins no, i també sobre alguns dels components químics i biològics que hi poden aparèixer. Aquesta part de recerca prèvia m’ha ajudat a tenir una base abans de començar a experimentar.



%%%%He puesto toda la informacion de abajo, en los subsecciones, mira a ver que tal esta la cosa%%%


%%%%% Aprofita el que tens per completar el que falta i si falta alguna cosa ho afegeixes.
%%%%%%%%%%%%%%%%%%%%%%%%%%%%%%%%%%%%%%%%%%%%%%%%%%%%%%%%%%%%%%%%%%%%%%%%%%
En aquest treball, l’objectiu principal és analitzar la qualitat de l’aigua mitjançant l’ús de tires reactives. Per a fer-ho de manera adequada, ha estat necessari dur a terme una recerca prèvia sobre els paràmetres de qualitat de l’aigua, els valors mínims i màxims recomanats i alguns dels molts components químics i biològics que poden estar-hi presents. Aquesta fase teòrica ha permès adquirir una base sòlida per a la experimentació.

A més, aquest treball no es limita a la recerca bibliogràfica i l’anàlisi de resultats, sinó que també ha implicat un procés d’aprenentatge personal. He hagut d’adaptar-me a l’ús de noves eines experimentals, proporcionades pel meu tutor de TR, Fernando García, i aprendre a aplicar-les correctament. Aquest procés metodològic es descriu amb més detall al capítol de Metodologia (vegeu capítol~\nameref{c:Metodologia}).

Com a objectiu personal, vull que aquest treball no només tingui un interès acadèmic, sinó que també sigui capaç de resultar interesant per a persones que habitualment no s’interessen per temes científics. Fer entenedor un tema tècnic com la qualitat de l’aigua és un repte personal que vull aconseguir.

Finalment, el treball de recerca també té un objectiu formatiu: aprendre a cercar informació de manera autònoma, a estructurar un treball científic i a realitzar una experimentació pròpia. Tot aquest procés em proporciona experiència i em prepara per a futurs reptes acadèmics i professionals.

