\chapter{RecercaPrevia}

\section{Què és la qualitat de l’aigua?}
La qualitat de l'aigua es un terme que es refereix a les caracteristicas fisicas, quimicas y bilogicas de l'aigua. Per determinar-la, s'analitzan alguns paràmetres.

Segons Wikipedia \cite{WikiAgua}, s'entén la qualitat de l'aigua com `característiques químiques, físiques, biològiques i radiològiques de l'aigua`, en relació amb necessitats humanes.

La Fundació Aquae \cite{Fundacionaquae} indica que es tracta d'un conjunt de paràmetres, com temperatura, contingut mineral i bacteris, mesurats i comparats amb estàndards, per definir si l'aigua és apta per a fins determinats o no.
\subsection{Tips de paràmetres per analitzar la qualitat de l'aigua}
\begin{enumerate}[1)]
 \item \textbf{Parametres químics}
 Relacionats amb la composició química de l’aigua; indiquen si és potable o contaminada.
 \begin{enumerate}[a)]
  \item \textit{pH}: Explicació a la secció \ref{subsec:ph}
  \item \textit{Duresa}: Explicació a la secció  \ref{subsec:duresa}
  \item \textit{Nitrats i Nitrits}: Explicació a la secció  \ref{subsec:nitratsnitrits}
  \item \textit{Clor (lliure i total)}: Explicació a la secció  \ref{subsec:clor}
  \item \textit{Metalls pesats}: Explicació a la secció   \ref{subsec:metallspesats}
 \end{enumerate}
 \item \textbf{Parametres físics}
 Mesuren característiques visibles o mesurables sense canviar la composició de l’aigua.
 \begin{enumerate}
  \item \textit{Temperatura}: Explicat a la secció \ref{subsec:temperatura}
  \item \textit{Color}: Explicat a la secció \ref{subsec:color}
  \item \textit{Olor i sabor}: Explicat a la secció \ref{subsec:olorisabor}
 \end{enumerate}
 \item \textbf{Parametres biologics}
 Indiquen la presència de microorganismes que poden ser patògens.
 \begin{enumerate}
  \item \textbf{Coliformes fecals}: Explicat a la secció \ref{subsec:coliformes}
  \item \textbf{Bacteris totals}: Explicat a la secció \ref{subsec:bacteris}
  \item \textbf{Algs i protozous}: Explicat a la secció \ref{subsec:algsiprotozous}
  \item \textbf{Índex biològic}: Explicat a la secció \ref{subsec:indexbiologic}

 \end{enumerate}

\end{enumerate}

\section{Explicació dels parametres químics}

\subsection{pH} \label{subsec:ph}
\subsubsection{Què és el pH?}
El pH és una mesura que serveix per establir el nivell d’acidesa o alcalinitat d'una dissolució. La `p` ve de `potencial` i l'`H` ve de l’àtom d’hidrogen, per això el pH és el potencial de l’hidrogen.

S'expressa com el logaritme negatiu de base 10 de la concentració de ions d'hidrogen: $ \text{pH} = -\log_{10} [\mathrm{H}^+] $.

A la fórmula la ${H}^+$ és la concentració de ions d'hidrogen en la solució, mesurat en mols per litre (mol/L). $-\log_{10}$ és el logaritme en base 10, el signe negatiu s'utilitza amb l'objectiu que el pH sigui un número positiu. La raó és perquè el logaritme d'un número menor que 1 és negatiu.

D'altra part, el \textbf{pOH} és una mesura de concentració de ions hidroxil ${OH}^-$ en una dissolució. S'expressa com el logaritme negatiu de base 10 de la concentració de ions hidroxil, i a diferència del pH, s'utilitza per mesurar el nivell d’alcalinitat d'una dissolució. Es calcula amb la formula  $ \text{pOH} = -\log_{10} [\mathrm{OH}^-] $.
\subsubsection{Quina relacio hi ha entre el nivell dacidez y el pH?}
Les dissolucions àcides tenen una alta quantitat de ions d'hidrogen. Això significa que tenen baixos valors de pH, i per tant, el seu nivell d'acidesa és elevat. Així que una dissolució serà més o menys àcida depenent de la quantitat d'hidrogen que contingui aquesta.

D'altra banda, les dissolucions bàsiques (alcalines) tenen baixes quantitats de ions d'hidrogen. Això vol dir que tenen alts valors de pH, i per tant el seu nivell d'acidesa és baix.

\subsection{Duresa} \label{subsec:duresa}

\subsection{Nitrats i Nitrits} \label{subsec:nitratsnitrits}

\subsection{Clor} \label{subsec:clor}

\subsection{Metalls pesats} \label{subsec:metallspesats}


\section{Explicació dels parametres físics}

\subsection{Temperatura} \label{subsec:temperatura}

\subsection{Color} \label{subsec:color}

\subsection{Olor i sabor} \label{subsec:olorisabor}

\section{Explicació dels parametres biologics}

\subsection{Coliformes fecals} \label{subsec:coliformes}

\subsection{Bacteris totals} \label{subsec:bacteris}

\subsection{Algs i protozous} \label{subsec:algsiprotozous}

\subsection{Índex biològic} \label{subsec:indexbiologic}


