\chapter{RecercaPrevia}

\section{Què és la qualitat de l’aigua?}
La qualitat de l'aigua es un terme que es refereix a les caracteristicas fisicas, quimicas y bilogicas de l'aigua. Per determinar-la, s'analitzan alguns paràmetres.

Segons Wikipedia \cite{WikiAgua}, s'entén la qualitat de l'aigua com "característiques químiques, físiques, biològiques i radiològiques de l'aigua", en relació amb necessitats humanes.

La Fundació Aquae \cite{Fundacionaquae} indica que es tracta d'un conjunt de paràmetres, com temperatura, contingut mineral i bacteris, mesurats i comparats amb estàndards, per definir si l'aigua és apta per a fins determinats o no.
\subsection{Tips de paràmetres per analitzar la qualitat de l'aigua}
\begin{enumerate}[1)]
 \item \textbf{Parametres químics}
 Relacionats amb la composició química de l’aigua; indiquen si és potable o contaminada.
 \begin{enumerate}[a)]
  \item \textit{pH}: Es un dels paràmetres més importants. El pH mesura cuan acida o cuan alcalina es l'aigua amb la concentració d’ions d’hidrogen (H$^+$).El pH neutre és 7; aigua potable ha d’estar entre 6,5 i 8,5.
  \item \textit{Duresa}: Està relacionada amb el contingut en dissolució de cations alcalinoterris, principalment calci i magnesi.
  \item \textit{Nitrats i Nitrits}: El nitrat \(\mathrm{NO_3^-}\) i el nitrit \(\mathrm{NO_2^-}\) són components químics que contenen nitrogen i oxigen, són molt comuns en la naturalesa, l’agricultura i la indústria. Alts nivells poden ser tòxics, especialment per a nadons.
  \item \textit{Clor (lliure i total)}:El clor es un gas molt reactiu y toxic en la seva forma puta en l'aigua s'utilitza principalment per eliminar bacterias, virus y altres microorganismes. El clor lliure es responsable de la desinfeccio, el clor total, inclou el clor tant el lliure com el combinat. \cite{WikiClor}
  \item \textit{Metalls pesats}: Un metall pesant és un membre d'un grup d'elements químics no gaire definits que exhibeixen propietats metàl·liques com:
   \begin{enumerate}
    \item Metall de transició: Els metalls de transició són aquells elements químics que es troben a la part central del sistema periòdic, en el bloc constant D
    \item Semimetal: Es caracteritzen per un comportament intermedi entre metalls i no metalls, compartint característiques de tots dos. Com a regla general, i en la majoria dels casos, tendeixen a reaccionar químicament amb els no metàl·lics
   \end{enumerate}
   Alguns exemples de metalls pesats son el Ferro (Fe), el Plom (Pb), el Mercuri (Hg) i el Coure (Cu). \cite{WikiMetales}
  \item \textit{Oxigen dissolt (OD)}: Fonamental per la vida aquàtica; nivells baixos poden indicar contaminació.
 \end{enumerate}

 \item \textbf{Parametres físics}
 Mesuren característiques visibles o mesurables sense canviar la composició de l’aigua.
 \begin{enumerate}
  \item \textit{Temperatura}: Afecta la solubilitat de gasos (com l’oxigen) i l’activitat dels organismes aquàtics.
  \item \textit{Color}: Pot indicar presència de substàncies dissoltes, com metalls o matèria orgànica.
  \item \textit{Olor i sabor}: Poden indicar contaminació per matèria orgànica, clor o compostos químics.
 \end{enumerate}



 \item \textbf{Parametres biologics}
 Indiquen la presència de microorganismes que poden ser patògens.
 \begin{enumerate}
  \item
 \end{enumerate}

\end{enumerate}



