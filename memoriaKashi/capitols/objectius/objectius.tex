\chapter{Objectius}

En aquest treball, l’objectiu principal és analitzar la qualitat de l’aigua mitjançant l’ús de tires reactives. Per a fer-ho de manera adequada, ha estat necessari dur a terme una recerca prèvia sobre els paràmetres de qualitat de l’aigua, els valors mínims i màxims recomanats i alguns dels molts components químics i biològics que poden estar-hi presents. Aquesta fase teòrica ha permès adquirir una base sòlida per a la experimentació.

A més, aquest treball no es limita a la recerca bibliogràfica i l’anàlisi de resultats, sinó que també ha implicat un procés d’aprenentatge personal. He hagut d’adaptar-me a l’ús de noves eines experimentals, proporcionades pel meu tutor de TR, Fernando García, i aprendre a aplicar-les correctament. Aquest procés metodològic es descriu amb més detall al capítol de Metodologia (vegeu capítol~\nameref{c:Metodologia}).

Com a objectiu personal, vull que aquest treball no només tingui un interès acadèmic, sinó que també sigui capaç de resultar interesant per a persones que habitualment no s’interessen per temes científics. Fer entenedor un tema tècnic com la qualitat de l’aigua és un repte personal que vull aconseguir.

Finalment, el treball de recerca també té un objectiu formatiu: aprendre a cercar informació de manera autònoma, a estructurar un treball científic i a realitzar una experimentació pròpia. Tot aquest procés em proporciona experiència i em prepara per a futurs reptes acadèmics i professionals.

