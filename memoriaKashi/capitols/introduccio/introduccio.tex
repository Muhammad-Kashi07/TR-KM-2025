\chapter{Introducció}

L’aigua és un recurs fonamental per a la vida. Cada dia la fem servir per beure, cuinar, netejar o regar, i sovint donem per fet que sempre serà accessible i segura. Tanmateix, la seva qualitat no és un aspecte trivial: segons dades de l’Organització Mundial de la Salut \cite{OrgaMS}, més de 2.000 milions de persones al món consumeixen aigua contaminada, la qual cosa pot provocar greus problemes de salut i afectar els ecosistemes aquàtics.

Aquest treball de recerca (TR) es dividirà en 3 \textbf{capítols essencials}:

\begin{itemize}
  \item En el \textbf{Capítol 4}~\nameref{c:Metodologia}, em centraré en explicar les eines que he utilitzat per poder completar correctament aquest treball, així com les raons del seu ús.

  \item El \textbf{Capítol 3}~\nameref{c:RecercaPrevia} es centra en l’estudi de diversos paràmetres físics, químics i biològics que permeten determinar la qualitat de l’aigua. A través d’aquests indicadors (com el pH, la duresa, la presència de nitrats, el clor, la temperatura o la presència de microorganismes) és possible entendre millor fins a quin punt l’aigua és apta per al consum humà i per al manteniment dels ecosistemes.

  \item Finalment, en el \textbf{Capítol 5}~\nameref{c:Partpra}, desenvoluparé la part pràctica, on realitzaré un total de tres experiments amb tota la informació obtinguda de la recerca prèvia, per poder calcular la qualitat de l’aigua. Els resultats d’aquests experiments seran recollits i analitzats a les conclusions del treball. (\textbf{PON REFERENCIAS})
\end{itemize}

\textbf{EN ÀNGLES / IN ENGLISH}


