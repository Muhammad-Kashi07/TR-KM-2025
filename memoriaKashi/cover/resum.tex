\chapter*{Resum}
Aquest treball de recerca se centra en l'estudi de la qualitat de l'aigua, combinant recerca teòrica, experimentació pràctica i aprenentatge personal. L'objectiu principal ha estat comprendre els paràmetres que defineixen la qualitat de l'aigua i aprendre a mesurar-los de manera senzilla però efectiva mitjançant l'ús de tires reactives.

El procés ha implicat una recerca prèvia sobre els components químics i biològics presents a l'aigua i els valors recomanats per a cada paràmetre, així com l'aprenentatge de noves eines de recerca i documentació com Linux, GitHub i LaTeX. A la part pràctica, s'han realitzat experiments controlats amb mostres d'aigua per analitzar els resultats obtinguts i generar recomanacions sobre l'ús correcte de les tires reactives per obtenir mesures fiables.

Aquest treball no només permet comprendre millor la qualitat de l'aigua, sinó que també reflecteix un aprenentatge personal i formatiu: l'adquisició d'habilitats per buscar informació de manera autònoma, organitzar un treball científic i comunicar els resultats de manera clara i accessible, tant per a persones amb coneixements previs com per a aquelles sense experiència en química.

\subsubsection{En Àngles}
This research project focuses on the study of water quality, combining theoretical research, practical experimentation, and personal learning. The main goal has been to understand the parameters that define water quality and to learn how to measure them in a simple yet effective way using reactive strips.

The process involved prior research on the chemical and biological components present in water and the recommended values for each parameter, as well as learning to use research and documentation tools such as Linux, GitHub, and LaTeX. In the practical part, controlled experiments were conducted with water samples to analyze the results and provide recommendations on the correct use of reactive strips to obtain reliable measurements.

This work not only helps to better understand water quality but also reflects personal and educational growth: acquiring skills to search for information independently, organize a scientific report, and communicate results clearly and accessibly, both for people with prior knowledge and for those without experience in chemistry.
