\chapter{Introducció}
\label{c:intro}
% L’aigua és un recurs fonamental per a la vida. Cada dia la fem servir per beure, cuinar, netejar o regar, i sovint donem per fet que sempre serà accessible i segura. Tanmateix, la seva qualitat no és un aspecte trivial: segons dades de l’Organització Mundial de la Salut \cite{OrgaMS}, més de 2.000 milions de persones al món consumeixen aigua contaminada, la qual cosa pot provocar greus problemes de salut i afectar els ecosistemes aquàtics.
%
% Aquest treball de recerca (TR) es dividirà en 3 \textbf{capítols essencials}:
%
% \begin{itemize}
%   \item En el \textbf{Capítol 4}~\nameref{c:Metodologia}, em centraré en explicar les eines que he utilitzat per poder completar correctament aquest treball, així com les raons del seu ús.
%
%   \item El \textbf{Capítol 3}~\nameref{c:RecercaPrevia} es centra en l’estudi de diversos paràmetres físics, químics i biològics que permeten determinar la qualitat de l’aigua. A través d’aquests indicadors (com el pH, la duresa, la presència de nitrats, el clor, la temperatura o la presència de microorganismes) és possible entendre millor fins a quin punt l’aigua és apta per al consum humà i per al manteniment dels ecosistemes.
%
%   \item Finalment, en el \textbf{Capítol 5}~\nameref{c:Partpra}, desenvoluparé la part pràctica, on realitzaré un total de tres experiments amb tota la informació obtinguda de la recerca prèvia, per poder calcular la qualitat de l’aigua. Els resultats d’aquests experiments seran recollits i analitzats a les conclusions del treball. (\textbf{PON REFERENCIAS})
% \end{itemize}

L’aigua és un element essencial per a la vida, i la seva qualitat té un impacte directe sobre la salut humana i el medi ambient. Entendre com es poden mesurar els paràmetres que la defineixen és clau per prendre decisions conscients i per apreciar la importància de preservar aquest recurs tan valuós. Aquest treball de recerca s’ha centrat en analitzar la qualitat de l’aigua combinant teoria, pràctica i experiència personal, utilitzant un mètode senzill però efectiu: les tires reactives.

L’elecció d’aquest tema respon a la curiositat per descobrir de manera directa què determina la qualitat de l’aigua i com es poden interpretar els resultats d’una forma clara i entenedora. Al mateix temps, el treball ha permès aprendre a utilitzar eines i recursos que van més enllà de la recerca específica sobre l'aigua. Linux, GitHub i \LaTeX~permeten desenvolupar habilitats com l’organització, la planificació, la creació i la transferència de coneixement científic.

Aquest enfocament no només aporta coneixements sobre química i medi ambient, sinó que també mostra com es desenvolupa un procés d’investigació complet, des de la recerca bibliogràfica fins a l’anàlisi de resultats experimentals. L’objectiu és fer que el treball sigui comprensible per a tothom, independentment del nivell de coneixements previs, i alhora reflectir l’aprenentatge i l’experiència personal que he adquirit durant tot el procés.

\section{Motivació}

La principal motivació per triar aquest tema és poder comprendre i experimentar de primera mà com es pot avaluar la qualitat de l’aigua. Aquest projecte també ha servit per desenvolupar competències importants per als estudis futurs i per al creixement personal:

\begin{itemize}
    \item Aprendre a buscar informació de manera autònoma i estructurada, combinant recerca teòrica i pràctica.
    \item Familiaritzar-se amb eines professionals i recursos digitals que faciliten l’organització i el seguiment del treball.
    \item Millorar la capacitat de comunicar resultats de manera clara i accessible, tant per a persones amb coneixements previs com per a novells en química.
    \item Experimentar directament amb materials i tècniques de laboratori per entendre millor els conceptes teòrics.
\end{itemize}

\section{Estructura de la memòria}

Després d’aquesta introducció, al capítol~\ref{c:objectius}:~\nameref{c:objectius}, es presenten els objectius generals i específics del treball. La metodologia aplicada, el material de laboratori utilitzat i les tècniques pràctiques desenvolupades es descriuen al capítol~\ref{c:Metodologia}:~\nameref{c:Metodologia}. Tot seguit, al capítol~\ref{c:RecercaPrevia}:~\nameref{c:RecercaPrevia}, s’exposen els coneixements previs necessaris sobre la qualitat de l’aigua i els paràmetres que s’analitzen. A continuació es presenta la~\nameref{c:Partpra} al capítol~\ref{c:Partpra}. Els resultats experimentals obtinguts es presenten a la secció de~\ref{s:resultats}:~\nameref{s:resultats}. Finalment, el capítol~\ref{c:conclusions}:~\nameref{c:conclusions} ofereix les conclusions i possibles línies de recerca futura derivades del treball.



